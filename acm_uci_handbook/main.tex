% ACM@UCI Handbook
%
% Organizational Notes
% 3 Sections: Introduction, Data Structures, Algorithms
% Data Structures and Algorithms organized into category
%   These categories are sorted alphabetically
% Within each category, subsections are sorted alphabetically




\documentclass[letterpaper,11pt,twoside]{article}

\usepackage[utf8]{inputenc}
\usepackage[margin=1in]{geometry}
\usepackage{graphicx}
\usepackage{svg}
\usepackage{indentfirst}
\usepackage{setspace}
\usepackage{fancyhdr}
\usepackage{listings}
\usepackage{inconsolata}
\usepackage{amsmath}
\usepackage{amsfonts}

\title{ACM@UCI Handbook}
\author{Ryan Yoshida}

\fancyhead{}
\fancyfoot{}
\fancyhead[RO,LE] {\slshape \rightmark}
\fancyhead[LO,RE] {\slshape ACM@UCI Handbook}
\fancyfoot[C] {\thepage}

\definecolor{codeblockbg}{rgb}{0.95,0.95,0.92}
\definecolor{codeblockcomment}{rgb}{0.75,0.75,0.75}
\lstdefinestyle{codeblock} {
    backgroundcolor=\color{codeblockbg},
    commentstyle=\color{codeblockcomment},
    basicstyle=\ttfamily\small,
    xleftmargin=\fboxsep,
    xrightmargin=-\fboxsep
}

\lstset{style=codeblock}

\newcommand{\PartDivider}[1] {
    \clearpage
    \thispagestyle{empty}
    \vspace*{\stretch{2}}
    \begin{center}
        \part{#1}
    \end{center}
    \vspace*{\stretch{4}}
    \clearpage
}

\begin{document}
    
\begin{titlepage}

    \thispagestyle{empty}
    \raggedleft
    \rule{1pt}{\textheight}
    \hspace{0.05\textwidth}
    \parbox[b]{0.75\textwidth}{
        
        {\Huge\bfseries ACM@UCI Handbook}\\[2\baselineskip]
        {\large\textit{ICPC Templates}}

        \vspace{0.1\textheight}
        
        \includesvg[width=0.4\columnwidth]{acm_uci_black.svg}\\[4\baselineskip]
        {\Large\textsc{Ryan Yoshida}}
    }
\end{titlepage}

    
    \setlength{\parindent}{0.5in}
    \singlespace
    \pagestyle{fancy}

    \tableofcontents

    \PartDivider{Introduction}
        \section{About}
            \subsection{Purpose}
            The ACM@UCI Handbook is a printable resource intended for use in contests such as ICPC.
            It provides data structure templates and code snippets for quick and reliable implementation.
            Priority is placed on data structures and algorithms which meet one or several of the following criteria: commonly seen in contests, difficult/tedious to remember/implement, and/or too niche to find in other competitive programming handbooks.

            This handbook is not intended to be used as a teaching resource.
            Explanations of underlying theory and concepts for the provided data structures and algorithms will be provided only insofar as it is expedient to their implementation in contest.
            To make the best use of this handbook you should first have a strong understanding the purpose and theory behind every data structure and algorithm.

            \subsection{Organization}


    \PartDivider{Data Structures}
        \section{Graphs}
            \subsection{Link-Cut Tree}
            \subsection{Splay Tree}
            \subsection{Union Find}

        \section{Range Queries}
            \subsection{Segment Tree}
                \subsubsection{Point Update}
                \subsubsection{Lazy Propagation}
            \subsection{Square Root Decomposition}
            \subsection{Sparse Table}
                Sparse tables are data structures which answer range queries in $O(1)$.
                Its construction takes $O(N\log N)$ and it requires $O(N\log N)$ space.
                It only works if the operation is idempotent (i.e. for operation $*$: $a*a = a$).
                Examples of idempotent functions include \emph{min}, \emph{gcd}.
                
                The provided C++ implementation takes an IdempotentFunc as a template type.
                Pass a type with an implemented functor for this.
                \lstinputlisting[language=C++]{snippets/sparse_table.cpp}
            \subsection{Wavelet Tree}

        \section{String Processing}
            \subsection{Aho Corasick}
                Aho Corasick creats a finite automata on a prefix trie (\ref{trie}) to match occurences of a set of strings in a search string.
                Equivalently, it can be seen as an implementation of Knuth Morris Pratt (\ref{kmp}) for a set of strings rather than one pattern string.
                It works the same as KMP by creating pointers to the longest proper suffix that is a prefix for each index of each string in the set.
                Construction of the Aho Corasic Trie is $O(M)$ where $M$ is the sum of character counts of all pattern strings.
                Iteration through the search string is worst case $O(NK)$ where $N$ is the length of the search string and $K$ is the number of pattern strings.
                In most cases however, it is safe to assume that iteration through the search string is $O(N)$.

                \textbf{Implementation:} The Aho Corasick Trie is encapsulated in a class.
                Due to the nature of the Aho Corasick automata construction algorithm, an \verb|Aho| object must be constructed with a vector of pattern strings \verb|dict|.
                To traverse the Aho Corasick Trie, a helper class \verb|Aho::AhoAutomata| is provided.
                The \verb|AhoAutomata| represents a state (node) in the Aho Corasick Trie.
                An iterator is provided for the \verb|AhoAutomata| to iterate through the \emph{ids} of all pattern strings in \verb|dict| that are terminated at that node (the \emph{id} of a pattern string is its index in \verb|dict|).
                Note that the \verb|Node| struct stores the child pointers as an \verb|std::array| of length 256.
                This can be adjusted based on the problem.
                \lstinputlisting[language=C++]{snippets/aho_corasick.cpp}

                \textbf{Example Usage:} An example of how to use the \verb|Aho| class is provided below.
                \lstinputlisting[language=C++]{snippets/aho_corasick_example.cpp}
            \subsection{EerTreE}
                EerTreE or Palindromic Tree is a data structuring for finding and tracking palindromes in a string.
                It is similar to the Aho Corasick tree as it uses suffix links to the longest proper suffix that is a palindrome.
                This implementation provides an iterator that, at each character, iterates through the lengths of all palindromes ending at that character.
                Construction of the EerTreE is $O(N)$.
                \lstinputlisting[language=C++]{snippets/EerTreE.cpp}
            \subsection{Suffix Array}
                A suffix array sorts the suffixes of a string in lexicographical order.
                The below provided implementation constructs a suffix array in $O(N\log N)$.
                Also included in this implementation is the construction of the longest common prefix array which gives the longest common prefix between suffixes adjacent in the suffix array.
                The \verb|LCP| array is constructed in $O(N)$.
                \lstinputlisting[language=C++]{snippets/suffix_array.cpp}
            \subsection{Trie} \label{trie}
                A Trie (short for reTRIEval tree), or Prefix Tree, stores a dictionary of strings in a manner that makes lookup by their prefix very efficient.
                Construction is $O(N)$ where $N$ is the sum of character counts of all dictionary strings.
                Lookup of a word or prefix is $O(M)$ where $M$ is the length of the query string.
                \lstinputlisting[language=C++]{snippets/trie.cpp}


    \PartDivider{Algorithms}
        \section{Geometry}
            \subsection{Closest Points}
            \subsection{Convex Hull}

        \section{Graphs}
            \subsection{Bellman-Ford's Algorithm: Single Source Shortest Path}
            \subsection{Dijkstra's Algorithm: Single Source Shortest Path}
            \subsection{Floyd-Warshall's Algorithm: All Pairs Shortest Path}
                Floyd-Warshall's Algorithm computes the shortest path between every pair of nodes in a graph.
                Acting on an adjacency matrix \verb|adj| where \verb|adj[u][v]| is the weight of an edge from \verb|u| to \verb|v|, it fills the matrix so that \verb|adj[u][v]| contains the cost of the shortest path between vertices \verb|u| and \verb|v|.
                This is done in $O(N^{3})$.

                For implementation ensure that the adjacency matrix is filled in properly.
                Be sure to consider the case of self-loops: should they be initialized to \verb|0|?
                Be sure to consider double edges (more than one edge between two vertices): should you take the minimum?
                Are there negative edge weights?
                This may result in negative cycles.
                A negative cycle will result in \verb|adj[u][u] < 0| after Floyd Warshall's is complete.

                \lstinputlisting[language=C++]{snippets/floyd_warshalls.cpp}
                \lstinputlisting[language=Python]{snippets/floyd_warshalls.py}
            \subsection{Minimum Spanning Tree}
                \subsubsection{Kruskall's Algorithm}
                \subsubsection{Prim's Algorithm}
            \subsection{Network Flow}
                \subsubsection{Edmonds Karp's Algorithm}
                \subsubsection{Dinic's Algorithms}
                \subsubsection{Hopcroft Karp's Algorithm}
                \subsubsection{Min-Cost Max Flow}
            \subsection{Tarjan's Algorithm}
                \subsubsection{Bridges \& Articulation Points}
                \subsubsection{Strongly Connected Components}
            \subsection{Trees}
                \subsubsection{Binary Lift}
                \subsubsection{Lowest Common Ancestor}

        \section{Math}
            \subsection{Arithmetic Sum}
                Computes an arithmetic sum constant time.
                The function was intentionally designed to have a similar function signature to Python's \verb|range| function.
                Thus, \verb|arithmetic_sum(start,end,step)| will yield the same result as \verb|sum(range(start,end,step))| in Python, but it will do so in $O(1)$.
                For this function to work, $start < end$ and $step > 0$.

                \textbf{Important:} \verb|start| is inclusive, \verb|end| is exclusive.
                \lstinputlisting[language=C++]{snippets/arithmetic_sum.cpp}
                \lstinputlisting[language=Python]{snippets/arithmetic_sum.py}
            \subsection{Chinese Remainder Theorem}
            \subsection{Division Over Modulo}
                Computes $\frac{a}{b} \mod mod$ by multiplying \verb|a| to the inverse mod of \verb|b|.
                Uses the power over modulo function described in \ref{power}.
                This function only works if \verb|b| and \verb|mod| are relatively coprime (i.e. $gcd(b,mod) = 1$).
                Can be used to compute inverse modulo by setting \verb|a=1|.
                The time complexity of this algorithm is $O(\log_2{mod})$.
                \lstinputlisting[language=C++]{snippets/division.cpp}
                \lstinputlisting[language=Python]{snippets/division.py}
            \subsection{Fast Fourier Transform}
            \subsection{Geometric Sum}
                \subsubsection{Finite Bounds}
                    Returns $\sum_{k=start}^{end-1}a^{k} = \frac{a^{start} - a^{end}}{1 - a}$.
                    \lstinputlisting[language=C++]{snippets/geometric_sum_finite.cpp}
                    \lstinputlisting[language=Python]{snippets/geometric_sum_finite.py}
                \subsubsection{Finite Start, Infinite End}
                    Returns $\sum_{k=start}^{\infty}a^{k} = \frac{a^{start}}{1 - a}$.
                    \lstinputlisting[language=C++]{snippets/geometric_sum_infinite.cpp}
                    \lstinputlisting[language=Python]{snippets/geometric_sum_infinite.py}
                \subsubsection{Infinite Scaled Sum}
                    Returns $\sum_{k=0}^{\infty}ka^{k} = \frac{a}{(1-a)^{2}}$
                    \lstinputlisting[language=C++]{snippets/geometric_sum_scaled.cpp}
                    \lstinputlisting[language=Python]{snippets/geometric_sum_scaled.py}
            \subsection{Greatest Common Divisor} \label{gcd}
                The algorithm for finding greatest common divisor is Euclid's algorithm.
                The time complexity of this algorithm is $O(\log_{10} max(a,b))$.
                \lstinputlisting[language=C++]{snippets/gcd.cpp}
                \lstinputlisting[language=Python]{snippets/gcd.py}
            \subsection{Lowest Common Multiple}
                The algorithm for finding lowest common multiple uses Euclid's algorithm for greatest common divisor (\ref{gcd}).
                The time complexity of this algorithm is $O(\log_{10} max(a,b))$.
                \lstinputlisting[language=C++]{snippets/lcm.cpp}
                \lstinputlisting[language=Python]{snippets/lcm.py}
            \subsection{Modulo} \label{modulo}
                A safe modulo for use with negative numbers.
                Use for languages like C++ when \verb|a| can be negative.
                \lstinputlisting[language=C++]{snippets/modulo.cpp}
            \subsection{Power Over Modulo} \label{power}
                Returns $b^{k} \mod mod$.
                This implementation cannot handle $b < 0$ in C++.
                To handle this case, use the safe modulo described in section \ref{modulo}.
                Note that no Python implementation is provided because it is already implemented in the standard library function \verb|pow(a,b[,mod])|.
                The time complexity of this algorithm is $O(\log_{2}(k))$.
                \lstinputlisting[language=C++]{snippets/power.cpp}
        \section{Strings}
            \subsection{Knuth Morris Pratt: String Matching} \label{kmp}
\end{document}
